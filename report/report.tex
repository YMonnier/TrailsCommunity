\documentclass[titlepage]{report}

\usepackage[utf8]{inputenc}
\usepackage[T1]{fontenc}
\usepackage[francais]{babel}
\usepackage{listings} %listings for code
\usepackage{graphicx} % can modify title style
\usepackage{titling} % can use pretitle to have a picture on title page
\usepackage{chngcntr} % permits to reset chapter numerization in each part
\usepackage{hyperref} % makes table of contents and list of figures clickable 
\usepackage{makecell} % can create newline in a table cell easily

%\hypersetup{colorlinks,citecolor=black,filecolor=black,linkcolor=black,urlcolor=black}

\counterwithin*{chapter}{part}

\addtolength{\oddsidemargin}{-2cm}
\addtolength{\evensidemargin}{-2cm}
\addtolength{\textwidth}{4cm}
\addtolength{\topmargin}{-2cm}
\addtolength{\textheight}{4cm}

\setcounter{secnumdepth}{3}

% Title Page

\pretitle{
	\begin{center}
		\LARGE
		\includegraphics[width=\textwidth]{Images/TrailsCommunity_doc.png}\\[\bigskipamount]
	}
	
	\posttitle{\end{center}}

		\title{Rapport de projet - Assitant Offroad}
		
		\author{BELLANGER Stéphen \and
			MONNIER Ysée \and
			DESHORS Yann}
		\date{Novembre 2016 - Decembre 2016}
		


\begin{document}
	
\maketitle

\tableofcontents

\listoffigures 

\part{Client TrailsCommunity}

\chapter{Introduction}

\section{Introduction}
\par
L'objectif de ce projet est de réaliser une application Android d'assistant offroad en se focalisant sur sa conception UML. 

\par
L'application crée devra permettre d'enregistrer, de visualiser et de partager des positions GPS entre un groupe d'utilisateur. Cette application pourra donc être utilisé dans le cadre d'activité en groupe et en plein air(randonnées, painball, vélo). De plus, chaque utilisateur pourra visualiser différentes statistiques en rapport avec ses activités réalisées précédemment.
\par
Ce rapport présentera la conception et la réalisation de l'application mobile en respectant la chronologie de notre étude : spécifications, étude UML, présentation du résultat puis un retour sur le déroulement du projet.
\par D'un point de vue technique, l'application mobile sera développée en Java avec l'IDE Android Studio pour le client compatible uniquemment pour Android. Celui-ci communiquera avec un service web en Ruby(framework Ruby On Rails).

\section{Répartition du travail}

\par La majorité de la phase de conception a été réalisée à trois : les spécifications et la modélisation de l'axe fonctionnel ont été faites par le groupe au complet.
Les diagrammes de séquence et d'activité ont été répartis entre nous, puis nous les avons tous passé en revue et modifié jusqu'à que nous soyons tous d'accord sur l'ensemble des diagrammes et des idées qu'elles véhiculent.

\par Lors de la phase de développement, nous avons réparti les rôles de la manière suivante : \begin{itemize}
	\item Ysée : Conception UMl, développement du serveur, déveleoppement de l'application mobile
	\item Stéphen : Conception UML, développement de l'application mobile, rapport
	\item Yann : Conception UML, rapport
\end{itemize}
\par Toutefois, vu l'objectif du projet, nous avons tous participé au développement du client, mais à un niveau d'implication différent en fonction des compétences des membres du groupe.

\chapter{Analyse}
\section{Dictionnaires}
\subsection{Termes}


\paragraph{Sondage} Un sondage est un tableau regroupant différent composant date/horaire ainsi que les réponses des différents utilisateurs.
\paragraph{Lien} Un lien est un identifiant unique pour retrouver un sondage.
\paragraph{Administrateur} Un administrateur est un utilisateur avec des droits supplémentaires. Il peut accéder à la création d’un sondage, à la modification et suppression d’un sondage qu’il a créé.
\paragraph{Utilisateur} Un utilisateur est une personne qui utilise l’application B00DLE sans compte administrateur.
\paragraph{Date} Une date correspond à un ensemble date en aaaa-mm-jj et une horaire en hh:mm.

\subsection{Actions}
\paragraph{Diffuser} Envoyer un e-mail du lien du sondage à l’ensemble des différentes adresses.
\paragraph{Clôturer} Terminer le droit de répondre à un sondage.
\paragraph{Sélectionner} Cocher dans une case à cocher une date.
\paragraph{Générer} Créer un lien unique qui correspond à un sondage.


\chapter{Introduction}

\section{Objectifs}

//A refaire : mal expliqué
TrailsCommunity est une application Android développé en Java sous IDE Android Studio. Elle est composé d'un serveur développé en Ruby (Framework Ruby On Rails).
La porté de l'application est l'ensemble des utilisateurs ayant pour objectifs de réalisé des activités en pleines air. Nous estimons une tranche d'âge entre 18 à 80 ans.
Toutefois, elle peut aussi être utilisé par des professionelles ou des personnes voulant organisé ou suppervisé des sorties.
La version sera disponible sur le Google play gratuitement. 

$<$Identify the product whose software requirements are specified in this 
document, including the revision or release number. Describe the scope of the 
product that is covered by this SRS, particularly if this SRS describes only 
part of the system or a single subsystem.$>$

\section{Conventions}

Le logo de l'application souhaite représenter la nature est les activités en pleines airs. 
La typographie du nom TrailsCommunity est : ...

//Faire une charte graphique -> ajouter a la TODO liste

La liste des fonctionnalités et leurs priorités sont présentés plus tard dans ce rapport. Celle-ci 
ont été déterminé avec l'ensemble des avis des membres du groupe. Le classement est rélisé par une échelle
de 3 graduation : bas, moyen, haut.

$<$Describe any standards or typographical conventions that were followed when 
writing this SRS, such as fonts or highlighting that have special significance.  
For example, state whether priorities  for higher-level requirements are assumed 
to be inherited by detailed requirements, or whether every requirement statement 
is to have its own priority.$>$

\section{Intended Audience and Reading Suggestions}

Ce document est destiné dans un premier temps au correcteur désigné de l'Université de Toulon pour la correction de ce projet.
Cependant, il sera alors disponible en public dans le dépot GitHub public utilisé durant le développement de l'application.
L'ensemble de ce rapport est organisé de la manière suivant :
- la présentation du client, sa la pré-conception et de ca conception UML.
- puis vient la modélisation du serveur, ses conventions et l'ensemble de ses actions.
- pour finir  une annexe avec l'ensemble des diagrammes nécessaire au développement de l'ensemble de l'application qui n'on pu être présenté
dans les partie du client et du serveur.

$<$Describe the different types of reader that the document is intended for, 
such as developers, project managers, marketing staff, users, testers, and 
documentation writers. Describe what the rest of this SRS contains and how it is 
organized. Suggest a sequence for reading the document, beginning with the 
overview sections and proceeding through the sections that are most pertinent to 
each reader type.$>$

\section{Project Scope Description du projet}

L'objectif de ce projet est de réaliser une application Android d'assistant offroad.
L'application crée devra permettre d'enregistrer, de visualiser et de partager des positions GPS entre un groupe d'utilisateur. Cette application pourra donc être utilisé dans le cadre d'activité en groupe et en plein air(randonnées, painball, vélo). 
De plus, chaque utilisateur pourra visualiser différentes statistiques en rapport avec ses activités réalisées précédemment.
Il est plus que nécessaire pour ce genre d'application d'avoir un temps de réponse rapide du serveur. C'est pour cela que la stratégie
principal du groupe est de réalisé un serveur dédié uniquemment pour l'application.
De plus, les éxigences sur les régles de codage sont appliqué sur le client.

$<$Provide a short description of the software being specified and its purpose, 
including relevant benefits, objectives, and goals. Relate the software to 
corporate goals or business strategies. If a separate vision and scope document 
is available, refer to it rather than duplicating its contents here.$>$

\section{Références}

//Les références a ajoutés...
1. Le site du développeur android
2. La doc oracle java
3. AndroidHive
4. Le document que le prof nous a parlé de commentcamarche

$<$List any other documents or Web addresses to which this SRS refers. These may 
include user interface style guides, contracts, standards, system requirements 
specifications, use case documents, or a vision and scope document. Provide 
enough information so that the reader could access a copy of each reference, 
including title, author, version number, date, and source or location.$>$


\chapter{Description générale}

\section{Perspective du produit}

Est ce que c'est pertinent ? On en parle deja dans "Intended Audience and Reading Suggestions" et "Objectifs"

$<$Describe the context and origin of the product being specified in this SRS.  
For example, state whether this product is a follow-on member of a product 
family, a replacement for certain existing systems, or a new, self-contained 
product. If the SRS defines a component of a larger system, relate the 
requirements of the larger system to the functionality of this software and 
identify interfaces between the two. A simple diagram that shows the major 
components of the overall system, subsystem interconnections, and external 
interfaces can be helpful.$>$

\section{Fonctions du produit}

Meme chose ici. Voir -> "Project Scope Description du projet"

$<$Summarize the major functions the product must perform or must let the user 
perform. Details will be provided in Section 3, so only a high level summary 
(such as a bullet list) is needed here. Organize the functions to make them 
understandable to any reader of the SRS. A picture of the major groups of 
related requirements and how they relate, such as a top level data flow diagram 
or object class diagram, is often effective.$>$

\section{Classes et caractéristiques des utilisateurs}
$<$Identify the various user classes that you anticipate will use this product.  
User classes may be differentiated based on frequency of use, subset of product 
functions used, technical expertise, security or privilege levels, educational 
level, or experience. Describe the pertinent characteristics of each user class.  
Certain requirements may pertain only to certain user classes. Distinguish the 
most important user classes for this product from those who are less important 
to satisfy.$>$

\section{Environnement d'exploitation}
$<$Describe the environment in which the software will operate, including the 
hardware platform, operating system and versions, and any other software 
components or applications with which it must peacefully coexist.$>$

\section{Contraintes de conception et d'implémentation}
$<$Describe any items or issues that will limit the options available to the 
developers. These might include: corporate or regulatory policies; hardware 
limitations (timing requirements, memory requirements); interfaces to other 
applications; specific technologies, tools, and databases to be used; parallel 
operations; language requirements; communications protocols; security 
considerations; design conventions or programming standards (for example, if the 
customer’s organization will be responsible for maintaining the delivered 
software).$>$

\section{Documentation utilisateur}
$<$List the user documentation components (such as user manuals, on-line help, 
and tutorials) that will be delivered along with the software. Identify any 
known user documentation delivery formats or standards.$>$

\section{Hypothèses et dépendances}

$<$List any assumed factors (as opposed to known facts) that could affect the 
requirements stated in the SRS. These could include third-party or commercial 
components that you plan to use, issues around the development or operating 
environment, or constraints. The project could be affected if these assumptions 
are incorrect, are not shared, or change. Also identify any dependencies the 
project has on external factors, such as software components that you intend to 
reuse from another project, unless they are already documented elsewhere (for 
example, in the vision and scope document or the project plan).$>$


\chapter{Exigences de l'interface externe}

\section{Interfaces des utilisateurs}
$<$Describe the logical characteristics of each interface between the software 
product and the users. This may include sample screen images, any GUI standards 
or product family style guides that are to be followed, screen layout 
constraints, standard buttons and functions (e.g., help) that will appear on 
every screen, keyboard shortcuts, error message display standards, and so on.  
Define the software components for which a user interface is needed. Details of 
the user interface design should be documented in a separate user interface 
specification.$>$

\section{Interfaces logicielles}
$<$Describe the connections between this product and other specific software 
components (name and version), including databases, operating systems, tools, 
libraries, and integrated commercial components. Identify the data items or 
messages coming into the system and going out and describe the purpose of each.  
Describe the services needed and the nature of communications. Refer to 
documents that describe detailed application programming interface protocols.  
Identify data that will be shared across software components. If the data 
sharing mechanism must be implemented in a specific way (for example, use of a 
global data area in a multitasking operating system), specify this as an 
implementation constraint.$>$

\section{Interfaces de communication}
$<$Describe the requirements associated with any communications functions 
required by this product, including e-mail, web browser, network server 
communications protocols, electronic forms, and so on. Define any pertinent 
message formatting. Identify any communication standards that will be used, such 
as FTP or HTTP. Specify any communication security or encryption issues, data 
transfer rates, and synchronization mechanisms.$>$


\chapter{Caractéristiques du système}
$<$This template illustrates organizing the functional requirements for the 
product by system features, the major services provided by the product. You may 
prefer to organize this section by use case, mode of operation, user class, 
object class, functional hierarchy, or combinations of these, whatever makes the 
most logical sense for your product.$>$

\section{Fonction système 1}
$<$Don’t really say “System Feature 1.” State the feature name in just a few 
words.$>$

\subsection{Description et priorité}
$<$Provide a short description of the feature and indicate whether it is of 
High, Medium, or Low priority. You could also include specific priority 
component ratings, such as benefit, penalty, cost, and risk (each rated on a 
relative scale from a low of 1 to a high of 9).$>$

\subsection{Séquence de stimulation / réponse}
$<$List the sequences of user actions and system responses that stimulate the 
behavior defined for this feature. These will correspond to the dialog elements 
associated with use cases.$>$

\subsection{Exigences fonctionnelles}
$<$Itemize the detailed functional requirements associated with this feature.  
These are the software capabilities that must be present in order for the user 
to carry out the services provided by the feature, or to execute the use case.  
Include how the product should respond to anticipated error conditions or 
invalid inputs. Requirements should be concise, complete, unambiguous, 
verifiable, and necessary. Use “TBD” as a placeholder to indicate when necessary 
information is not yet available.$>$

$<$Each requirement should be uniquely identified with a sequence number or a 
meaningful tag of some kind.$>$

REQ-1:	REQ-2:

\section{System Feature 2 (and so on)}


\chapter{Autres exigences non fonctionnelles}

\section{Exigences de performance}
$<$If there are performance requirements for the product under various 
circumstances, state them here and explain their rationale, to help the 
developers understand the intent and make suitable design choices. Specify the 
timing relationships for real time systems. Make such requirements as specific 
as possible. You may need to state performance requirements for individual 
functional requirements or features.$>$

\section{Safety Requirements}
$<$Specify those requirements that are concerned with possible loss, damage, or 
harm that could result from the use of the product. Define any safeguards or 
actions that must be taken, as well as actions that must be prevented. Refer to 
any external policies or regulations that state safety issues that affect the 
product’s design or use. Define any safety certifications that must be 
satisfied.$>$

\section{Exigences de sécurité}
$<$Specify any requirements regarding security or privacy issues surrounding use 
of the product or protection of the data used or created by the product. Define 
any user identity authentication requirements. Refer to any external policies or 
regulations containing security issues that affect the product. Define any 
security or privacy certifications that must be satisfied.$>$

\section{Attributs de qualité du logiciel}
$<$Specify any additional quality characteristics for the product that will be 
important to either the customers or the developers. Some to consider are: 
adaptability, availability, correctness, flexibility, interoperability, 
maintainability, portability, reliability, reusability, robustness, testability, 
and usability. Write these to be specific, quantitative, and verifiable when 
possible. At the least, clarify the relative preferences for various attributes, 
such as ease of use over ease of learning.$>$

\chapter{Autres exigences}
$<$Define any other requirements not covered elsewhere in the SRS. This might 
include database requirements, internationalization requirements, legal 
requirements, reuse objectives for the project, and so on. Add any new sections 
that are pertinent to the project.$>$

\section{Annexe A: Glossaire}
%see https://en.wikibooks.org/wiki/LaTeX/Glossary
$<$Define all the terms necessary to properly interpret the SRS, including 
acronyms and abbreviations. You may wish to build a separate glossary that spans 
multiple projects or the entire organization, and just include terms specific to 
a single project in each SRS.$>$

\section{Annexe B: Modèles d'analyse}
$<$Optionally, include any pertinent analysis models, such as data flow 
diagrams, class diagrams, state-transition diagrams, or entity-relationship 
diagrams.$>$

\section{Annexe C: Liste à déterminer}
$<$Collect a numbered list of the TBD (to be determined) references that remain 
in the SRS so they can be tracked to closure.$>$

\end{document}          
